\documentclass[11pt, a4paper,english,spanish]{amsart}
     \parindent = 10 pt
        \parskip=1.5pt 
        \usepackage[width=15.5cm, left=3cm, top=2.5cm, height= 24.5cm]{geometry}

\usepackage[utf8]{inputenc} % para poder usar tildes en archivos UTF-8
\usepackage[spanish]{babel} % para que comandos como \today den el resultado en castellano
\usepackage{a4wide} % márgenes un poco más anchos que lo usual
\usepackage[]{caratula}


\usepackage{palatino, eulervm}
\linespread{1.05}        % Palatino needs more leading
\usepackage[section]{placeins} %floatings(as figures) remain in their section

% %% Useful packages
% \usepackage{amsmath}
% \usepackage{graphicx}
% \usepackage[colorinlistoftodos]{todonotes}
% \usepackage[colorlinks=true, allcolors=blue]{hyperref}

\begin{document}

\titulo{Trabajo Práctico Final}
\subtitulo{Modelo Navier Stokes 2D}

\fecha{\today}

\materia{Organización del Computador II}

\integrante{Ventura, Martín Alejandro}{249/11}{venturamartin90@gmail.com}
\integrante{Muiño, María Laura}{399/11}{mmuino@dc.uba.com}

\maketitle

%contenido 
\section{Introducción}
Los flujos son gobernados por ecuaciones diferenciales parciales, que representan las leyes de conservación de masa, momento y energía. La dinámica de fluidos computacional se encarga de resolver esas ecuaciones diferenciales utilizando técnicas de análisis numérico. Las computadoras son utilizadas para realizar los cálculos requeridos para simular la interacción entre líquidos, gases y superficies definidas por las condiciones de borde. Disponer de más poder computacional es útil para disminuir el tiempo requerido para realizar las simulaciones, o aumentar la calidad de los resultados.

~\\
Para poder aumentar el poder computacional disponible, se utilizan a menudo, técnicas de cómputo en paralelo, o de cómputo vectorial. En este trabajo nos centraremos en la tecnología de cómputo vectorial SIMD (Single Instruction Multiple Data) de Intel.

~\\
Concretamente se desarrollará código en assembler, que utilizando las instrucciones de vectorización de los procesadores Intel, logre un aumento de rendimiento. Luego se comparará ese aumento de rendimiento con técnicas automáticas de vectorización o paralelización, tales como OpenMP, una API para el procesamiento multinúcleo con memoria compartida, y las optimizaciones disponibles en los compiladores ICC (Intel C Compiler) y GCC (GNU Compiler Collection), que a su vez utilizan instrucciones SIMD.

~\\
Hay diversos problemas de flujo conocidos que son utilizados frecuentemente para testear aplicaciones de este estilo. En este trabajo utilizaremos cavity flow.

El problema conocido como Lid-Driven Cavity Flow ha sido largamente usado como caso de validación para nuevos códigos y métodos. La geometría del problema es simple y bidimensional, las condiciones de borde son también sencillas. El caso estándar consta de un fluido contenido en un dominio cuadrado con condiciones de borde de Dirichlet en todas las paredes, con tres lados estacionarios y un lado en movimiento (con velocidad tangente a la pared), que induce velocidad en el fluido.

\newpage

\section{Desarrolo}

\subsection{Discretización}
El problema de Navier Stokes es gobernado por la siguiente escuación

$$\frac{\partial \vec{v}}{\partial t}+(\vec{v}\cdot\nabla)\vec{v}=-\frac{1}{\rho}\nabla p + \nu \nabla^2\vec{v}$$


Los operadores presentes en la primera ecuación presentada, al ser usados en su forma bidimencional, permiten una reescritura como la siguiente:

$$\frac{\partial u}{\partial t}+u\frac{\partial u}{\partial x}+v\frac{\partial u}{\partial y} = -\frac{1}{\rho}\frac{\partial p}{\partial x}+\nu \left(\frac{\partial^2 u}{\partial x^2}+\frac{\partial^2 u}{\partial y^2} \right)$$

$$\frac{\partial v}{\partial t}+u\frac{\partial v}{\partial x}+v\frac{\partial v}{\partial y} = -\frac{1}{\rho}\frac{\partial p}{\partial y}+\nu\left(\frac{\partial^2 v}{\partial x^2}+\frac{\partial^2 v}{\partial y^2}\right)$$

$$\frac{\partial^2 p}{\partial x^2}+\frac{\partial^2 p}{\partial y^2} = -\rho\left(\frac{\partial u}{\partial x}\frac{\partial u}{\partial x}+2\frac{\partial u}{\partial y}\frac{\partial v}{\partial x}+\frac{\partial v}{\partial y}\frac{\partial v}{\partial y} \right)$$

Las primeras dos ecuaciones se corresponden con la velocidad en las direcciones en x e y, mientras que la tercera da cuenta de los efectos de la presión.

Comenzaremos con algunas definiciones. Al modelar con diferencias finitas, se utilizan ciertos reemplazos de los operadores diferenciales conocidos como discretizaciones. Como su nombre indica, estas son versiones discretas de los operadores, y se las usa bajo el supuesto de que en el límite se comportan de forma similar. Pasaremos ahora a definir algunas discretizaciones que serán utilizadas para modelar el problema.
~\\
~\\
\begin{minipage}{\linewidth}

Centradas de primer orden:
\begin{center}

~\\
$\frac{dU}{dx} = \frac{U^{n}_{i+1,j} - U^{n}_{i-1,j}}{2dx} $
~\\
~\\
$\frac{dU}{dy} = \frac{U^{n}_{i,j+1} - U^{n}_{i,j-1}}{2dy} $
~\\
~\\
$\frac{dU}{dt} = \frac{U^{n+1}_{i,j} - U^{n-1}_{i,j}}{2dt} $
~\\
\end{center}

\end{minipage}
\begin{minipage}{\linewidth}



Centradas de segundo orden:
\begin{center}

~\\
$\frac{d^{2}U}{dx^{2}} = \frac{ U^{n}_{i+1,j} - 2*U^{n}_{i,j} + U^{n}_{i-1,j}}{dx^2}$
~\\
~\\
$\frac{d^{2}U}{dy^{2}} = \frac{ U^{n}_{i,j+1} - 2*U^{n}_{i,j} + U^{n}_{i,j-1}}{dx^2}$
~\\
~\\
$\frac{d^{2}U}{dt^{2}} = \frac{ U^{n+1}_{i,j} - 2*U^{n}_{i,j} + U^{n-1}_{i,j}}{dt^2}$
~\\
\end{center}

\end{minipage}
\begin{minipage}{\linewidth}

Adelantadas de primer orden:
\begin{center}

~\\
$\frac{dU}{dx} = \frac{U^{n}_{i+1,j} - U^{n}_{i,j}}{dx} $
~\\
~\\
$\frac{dU}{dy} = \frac{U^{n}_{i,j+1} - U^{n}_{i,j}}{dy} $
~\\
~\\
$\frac{dU}{dt} = \frac{U^{n+1}_{i,j} - U^{n}_{i,j}}{dx} $
~\\
\end{center}

\end{minipage}
\begin{minipage}{\linewidth}


Atrasadas de primer orden:
\begin{center}

~\\
$\frac{dU}{dx} = \frac{U^{n}_{i,j} - U^{n}_{i-1,j}}{dx} $
~\\
~\\
$\frac{dU}{dy} = \frac{U^{n}_{i,j} - U^{n}_{i,j-1}}{dy} $
~\\
~\\
$\frac{dU}{dt} = \frac{U^{n}_{i,j} - U^{n-1}_{i,j}}{dx} $
~\\
\end{center}
\end{minipage}
~\\

\begin{minipage}{\linewidth}

Reemplazando estas discretizaciones en las ecuaciones semi-acopladas de Navier Stokes obtenemos: 
\begin{center}
~\\

~\\
$\frac{u_{i,j}^{n+1}-u_{i,j}^{n}}{\Delta t}+u_{i,j}^{n}\frac{u_{i,j}^{n}-u_{i-1,j}^{n}}{\Delta x}+v_{i,j}^{n}\frac{u_{i,j}^{n}-u_{i,j-1}^{n}}{\Delta y}
=-\frac{1}{\rho}\frac{p_{i+1,j}^{n}-p_{i-1,j}^{n}}{2\Delta x}+\nu (\frac{u_{i+1,j}^{n}-2u_{i,j}^{n}+u_{i-1,j}^{n}}{\Delta x^2}+\frac{u_{i,j+1}^{n}-2u_{i,j}^{n}+u_{i,j-1}^{n}}{\Delta y^2}) + Fu$
~\\
~\\
~\\

$\frac{v_{i,j}^{n+1}-v_{i,j}^{n}}{\Delta t}+u_{i,j}^{n}\frac{v_{i,j}^{n}-v_{i-1,j}^{n}}{\Delta x}+v_{i,j}^{n}\frac{v_{i,j}^{n}-v_{i,j-1}^{n}}{\Delta y}=-\frac{1}{\rho}\frac{p_{i,j+1}^{n}-p_{i,j-1}^{n}}{2\Delta y}
+\nu(\frac{v_{i+1,j}^{n}-2v_{i,j}^{n}+v_{i-1,j}^{n}}{\Delta x^2}+\frac{v_{i,j+1}^{n}-2v_{i,j}^{n}+v_{i,j-1}^{n}}{\Delta y^2}) + Fv$
~\\
~\\
~\\

$\frac{p_{i+1,j}^{n}-2p_{i,j}^{n}+p_{i-1,j}^{n}}{\Delta x^2}+\frac{p_{i,j+1}^{n}-2*p_{i,j}^{n}+p_{i,j-1}^{n}}{\Delta y^2} 
=\rho[\frac{1}{\Delta t}(\frac{u_{i+1,j}-u_{i-1,j}}{2\Delta x}+\frac{v_{i,j+1}-v_{i,j-1}}{2\Delta y})$
\end{center}

\end{minipage}
~\\
~\\

\begin{minipage}{\linewidth}
Aquí en la ultima ecuación podemos ver que no se reemplazó directamente cada operador mediante las ecuaciones de discretización, sino que se agregó un término temporal, sin que hubiera en principio información sobre el tiempo en la ecuación de la presión. Este cambio se hace con el objetivo de acoplar la ecuación de la presión con las ecuaciones de velocidad. El mecanismo por el cual la adición de este nuevo término acopla las ecuaciones, no se presentará en este trabajo.
~\\

Cabe aclarar que al discretizar, se puede modelar el sistema mediante un método implícito o explícito. Un método implícito, o parcialmente implícito, incluiría una ponderación entre los valores de las variables en la iteración n, y la iteración n+1. En este trabajo utilizaremos un método explicito, ya que el sistema de ecuaciones determinado por un método explicito es lineal, y resulta en relaciones donde un elemento en la iteración n+1 depende de otros en la iteración n, pudiendo entonces realizarse los reemplazos en las matrices que representan el sistema de forma directa, y resultando así en una implementación con menor dependencia de datos. Un método implícito da como resultado un sistema no lineal, en el cual hay que hacer uso de algún método de resolución de sistemas no lineales, como punto fijo, lo cual aumenta la complejidad de la implementación.

\end{minipage}

\newpage

\subsection{Implementación}
La implementación fue realizada completamente en C++, excepto por la sección donde es crítico el rendimiento, la cual fue programada en C++ y Assembler. Esta sección es la correspondiente a la función \textit{calcVelocities}, que como su nombre indica, calcula las velocidades en cada punto.
~\\
~\\
El programa define las matrices U1, U2, V1, V2, P1, P2, que representan el estado del sistema en una iteración para la velocidad en \textit{u}, en \textit{v}, y la presión, y luego estas mismas en la iteración siguiente. 
~\\
~\\
Se definen las condiciones iniciales del problema, y luego se utiliza un método explícito para calcular los nuevos valores del sistema. Estos son guardados en U2, V2, y P2. Seguido de esto, el programa reemplaza los valores de U1, V1, y P1, por aquellos de U2, V2 y P2, quedado así preparado para la siguiente iteración. 
~\\
~\\
Se implementó también una clase mat2, que representa una matriz, y que contiene un puntero a un arreglo de números de punto flotante de simple precisión y dos enteros que representan el tamaño en filas y columnas de la matriz. Además la clase cuenta con funciones que realizan la abstracción de indexar en el arreglo calculando la posición del elemento buscado como la columna pedida, más la fila pedida multiplicada por la cantidad de columnas. 
~\\
~\\
En cuanto a la vectorización, como se comentó anteriormente se utilizó la tecnología SIMD de Intel, de la forma descripta a continuación:
\begin{itemize}
	\item Mediante una directiva DEFINE presente en el Makefile, se elije si se desea compilar con soporte para SIMD, soporte para OpenMP, ambos, o ninguno.

	\item El programa define las matrices necesarias con los valores iniciales según lo estipulado por el método de discretización utilizado.

	\item La sección del programa que realiza el cálculo consta de tres ciclos consecutivos. El primero cicla en la variable \textit{t}, que representa el tiempo, el segundo en la variable \textit{i}, que representa la altura, y el tercero en la variable \textit{j} que representa el ancho.
	
    \item La paralelización mediante OpenMP se realiza en la variable \textit{i}.
	
    \item La vectorización mediante SIMD, se realiza en la variable \textit{j}. Es decir, en un solo llamado a la versión de Assembler de la función de cálculo se procesan 4 elementos consecutivos en memoria.

    \item Además, al utilizar SIMD, cuando se llega a un valor de \textit{j} menor al ancho de los registros XMM dividido por el tamaño del tipo de datos flotante de presición simple, se cambia el procesamiento mediante SIMD por el de C++, hasta que \textit{j} alcanza su valor máximo.

	\item Además, durante la simulación no se crean ni se destruyen matrices, sino que estas son reutilizadas cambiando los valores que contienen para no perder tiempo manejando memoria.
\end{itemize}
\newpage

\section{Conclusión}


Los resultados obtenidos sugieren que bien utilizado, mientras más específico es el método, mejor resultado produce, como puede verse en el aumento de rendimiento al pasar de GCC a ICC, y luego nuevamente de ICC a una versión escrita teniendo en cuenta la plataforma específica. Además se encontró evidencia que comprueba el poder de optimización de las distintas mejoras implementadas por los compiladores GCC e ICC, dando lugar a velocidades mucho mayores que las versiones originales. 
~\\
~\\
Se estudió también el procesamiento multi-núcleo, y se encontró que mejora el rendimiento por sobre lo que es posible mejorarlo utilizando optimizaciones de compilador. Sin embargo, al no disponer de muchas unidades de procesamiento, la vectorización fue una herramienta más satisfactoria a la hora de disminuir el tiempo de ejecución. Esto se debió mayormente a que en el caso de este estudio, y por el equipo utilizado para realizar la experimentación, ambos SIMD y OpenMP, eran capaces de computar 4 elementos en simultáneo, con la salvedad de que el mecanismo mediante el cual lo hace SIMD es más sencillo.
~\\
~\\
 Otra cosa que se pudo apreciar, es la alta varianza en los tiempos de ejecución de OpenMP. Creemos que esto se debe a que si bien no se utilizaron los equipos durante la experimentación, el scheduler debe atender todas las tareas, y al tener los 4 núcleos saturados, la versión de OpenMP es especialmente vulnerable a los efectos del sistema operativo.
~\\
~\\
Como trabajo a futuro se plantea por un lado combinar SIMD con OpenMP, o alguna otra tecnología que permita paralelización, como MPI (Message Passing Interface). Esperamos que esto de aumentos mucho mayores de rendimiento, y en particular en el caso de MPI, permita la construcción de una versión escalable. Por otro lado, seria interesante realizar el mismo estudio a otro tipo de problemas, con le objetivo de analizar en un rango más grande de aplicaciones el efecto de estas técnicas.
\newpage


\end{document}