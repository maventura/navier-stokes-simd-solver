\documentclass[11pt, a4paper,english,spanish]{amsart}
     \parindent = 10 pt
        \parskip=1.5pt 
        \usepackage[width=15.5cm, left=3cm, top=2.5cm, height= 24.5cm]{geometry}

\usepackage[utf8]{inputenc} % para poder usar tildes en archivos UTF-8
\usepackage[spanish]{babel} % para que comandos como \today den el resultado en castellano
\usepackage{a4wide} % márgenes un poco más anchos que lo usual
\usepackage[]{caratula}


\usepackage{palatino, eulervm}
\linespread{1.05}        % Palatino needs more leading
\usepackage[section]{placeins} %floatings(as figures) remain in their section

% %% Useful packages
% \usepackage{amsmath}
% \usepackage{graphicx}
% \usepackage[colorinlistoftodos]{todonotes}
% \usepackage[colorlinks=true, allcolors=blue]{hyperref}

\begin{document}

\titulo{Trabajo Práctico Final}
\subtitulo{Modelo Navier Stokes 2D}

\fecha{\today}

\materia{Organización del Computador II}

\integrante{Ventura, Martín Alejandro}{249/11}{venturamartin90@gmail.com}
\integrante{Muiño, María Laura}{399/11}{mmuino@dc.uba.com}

\maketitle

%contenido 
\section{Introducción}%Describe lo realizado en el trabajo practico.

En el presente informe se resuelve el sistema de ecuaciones de Navier-Stokes para el cálculo de las velocidades de un fluido. Las instancias particulares del problema que se analizan son cavity flow y channel flow. Se implementó la solución en c++ y en asembler.

Se realizó una solución iterativa de punto fijo, creo.


\newpage

\section{Desarrolo}%Describe cada una de las funciones que implementaron, respondiendo en profundidad cada una de las preguntas de los Experimentos. Para la descripcion de cada funcion deberan decir como opera una iteracion del ciclo de la funcion. Es decir, como mueven los datos a los registros, como los reordenan para procesarlos, las operaciones que se aplican a los datos, etc. Para esto pueden utilizar pseudocodigo, diagramas (mostrando graficamente el contenido de los registros XMM) o cualquier otro recurso que le sea util para describir la adaptacion del algoritmo al procesamiento simultaneo SIMD. No se debera incluir el codigo assembler de las funciones (aunque se pueden incluir extractos en donde haga falta). Las preguntas en cada ejercicio son una guıa para la confeccion de los resultados obtenidos. Al responder estas preguntas, se deberan analizar y comparar las implementaciones de cada funciones en su version C y ASM, mostrando los resultados obtenidos a traves de tablas y graficos. Tambien se debera comentar acerca de los resultados obtenidos. En el caso de que sucediera que la version en C anduviese mas rapidamente que su version ASM, justificar fuertemente a que se debe esto.



\subsection{Experimentacion(version borrador)}

Usando la herramienta objdump sobre los archivos objeto (.o) del código de c++, obtuvimos y analizamos el código ensamblado por el compilador.
Cosas que notamos 
\begin{itemize}
	\item Dentro de la función calcVelocities, hay llamados varios recursivos que en al version de c++ no suceden
	\item Hay consultas a memorias innecesarias (pide un valora memoria, que no se pisa y luego de varias operaciones, vuelve a pedirlo, habiendolo tenido cargado)
\end{itemize}
\newpage

\section{Conclusión}% Reflexi ́on final sobre los alcances del trabajo pr ́actico, la programaci ́on en el modelo SIMD a bajo nivel, problem ́aticas encontradas, y todo lo que consideren pertinente.
\newpage


\end{document}